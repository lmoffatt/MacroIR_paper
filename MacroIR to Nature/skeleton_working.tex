\documentclass{article}

\usepackage{graphicx} 
\usepackage{amsmath} 
\usepackage{amsfonts} 
\usepackage{hyperref}
\usepackage{caption}

\title{Bridging Kinetics and Structural Insights: Bayesian Modeling of P2X2 Activation}

\author{Author Name}

\begin{document}
	
	\maketitle
	\abstract{Ligand-activated receptors are essential for cellular signaling and represent significant therapeutic targets. Among these, the P2X receptor family, activated by ATP, plays a crucial role in various physiological processes, yet the molecular mechanisms governing its activation remain inadequately understood. Here, we demonstrate that ATP binding induces asymmetric effects on the subunits of the binding site, significantly lowering the activation barrier for the rotation of the Upper Body-associated subunit while having modest effect on the Lower Body-associated subunit. This rotation exerts distinct influences on the adjacent binding sites: at the Upper Body site, ATP binding energy increases with a pronounced reduction in the unbinding rate, whereas at the Lower Body site, binding energy remains largely unchanged, but the increased activation barrier leads to reduced binding and unbinding rates. We re-analyze previously published data from outside-out patch preparations of rP2X2 in response to ultra-short ATP pulses of increasing concentrations  using a novel Bayesian algorithm that leverages kinetic information from response fluctuations. We anticipate that our Bayesian kinetic analysis will be instrumental in validating and refining the activation model presented here. This framework can be extended to investigate mutations, partial agonists, and other subunits. Furthermore, the notion of one subunit acting as an anchor while another serves as an allosterically active component may be applicable to other ligand-activated receptors. These insights could pave the way for developing innovative pharmaceutical agents targeting P2X receptors and related signaling pathways.
	}
		
	\section{Introduction}
	\textbf{Comment:} The introduction should frame the problem, present the current state of knowledge, and emphasize the novelty of your work. Clearly state the gap your study fills and the objectives of your work.
	
	Ion channels are essential for cellular signaling, and their activation mechanisms are fundamental for understanding physiological processes and designing therapeutic interventions. While structural biology has provided detailed insights into the architecture of ion channels, few studies have attempted to map kinetic parameters back to structural data. Here, we address this gap by applying a novel Bayesian modeling framework to analyze the activation kinetics of P2X2, a ligand-gated ion channel. By incorporating aspects of molecular dynamics and energy landscapes, we provide new insights into the structural mechanisms that govern receptor activation.
	
	\subsection{Background}
	\textbf{Comment:} Provide context for the study, introducing the P2X2 receptor, its biological importance, and previous attempts at understanding its activation mechanism.
	
	P2X2 receptors are involved in neurotransmission and play a key role in cellular signaling. Previous studies have explored the kinetics of P2X2 activation using experimental techniques such as patch-clamp, but these studies have not directly linked kinetic data to structural information. Understanding the activation mechanism at the molecular level requires not only measuring macroscopic current responses but also considering the underlying structural changes that occur upon ligand binding and channel opening. 
	
	
	
	\subsection{Research Question and Objective}
	\textbf{Comment:} Clearly define the question your study aims to address and the specific objectives.
	
	This study seeks to answer two central questions: How can we model the kinetic activation of P2X2 with high accuracy? And how can we map these kinetic parameters to structural changes in the receptor? To answer these questions, we introduce a Bayesian framework that allows us to model the activation kinetics of P2X2 receptors and explore how these kinetics relate to the receptor's energy landscape and structural dynamics.
	
	\section{Methods}
	\textbf{Comment:} In this section, explain your experimental design, data analysis methods, and the mathematical models used. Be clear about the novelty and how the Bayesian approach contributes to the analysis.
	How can we 
	
	
	
	\subsection{Experimental Data Re-analysis}
	\textbf{Comment:} Describe the experimental data you re-analyzed, including the specific data set from the study you referenced.
	
	The experimental data used in this study were obtained from our previous work, which provides a unique dataset for the re-analysis of P2X2 activation. These data are the only published measurements of the response of a ligand-gated ion channel to ATP pulses shorter than the channel's response time, while simultaneously recording the actual time profile of the agonist application on the same patch. This dataset is ideal for testing new kinetic models, as it provides exceptional time-resolved data that allow for accurate model fitting. 
	The data comprise the response of outside out patch of human embryonic kidney (HEK) 293T cells 393 cell overexpressing P2X2 receptors. Patches were clamped at -60 mV with a DAGAN 3900 patch-clamp amplifier (Dagan). The patch clamp head stage was mounted on a Burleigh Model PCS-1000 with a PCS-250 patch clamp driver (EXFO Life Sciences). ATP (Sigma-Aldrich) was dissolved in the perfusion solution, which contained (in mM) 150 NaCl, 2 KCl, 1.3 MgCl2, 10 HEPES,	pH 7.4. We compensated for the chelation of Mg+2 by ATP by adding MgCl2 to our solutions such that all solutions contained 1 mM free Mg+2, as determined by the program Bound and Determined (Brooks and Storey, 1992). Currents were low-pass filtered at 10 kHz with a four-pole Bessel filter and digitized at 50 kHz (Digidata 1322A, Axon Instruments). All experiments were done at room temperature (22°C).
	Ultrafast solution exchanges were obtained after calibrating the response of the application pippete as described. 
	Application of high concentrations of ATP for hundreds of milliseconds resulted in desensitization. Furthermore, the responses of outside-out patches containing P2X2 receptors ran down over time. To deal with these complications, we tested patches no sooner than 1 min after the previous stimulus and sandwiched each test concentration between two normalizing stimuli (1 mM ATP for 10 ms). The amplitude of the test response was then normalized to the average of the two normalizing responses. With this paradigm, the variability in amplitude between successive normalizing responses (spaced 2 min apart) was $<10\% $ in most trials. Trials in which the amplitude of the normalizing responses changed by $>10\%$ were not included in the analysis.
	
	
	
	\subsection{Bayesian Modeling Framework}
	\textbf{Comment:} Describe the Bayesian algorithm you used to fit the kinetic model and how it integrates with the structural analysis.
	
	We developed a novel Bayesian framework, using the MacroIR algorithm, to fit kinetic models to the experimental data. The MacroIR approach approximates the likelihood function of macroscopic currents by considering the time-averaged response over intervals, effectively converting the problem into a standard Markov process. This allows us to estimate the posterior distributions of kinetic parameters, such as the rates of channel opening and closing, and interpret these parameters in terms of the receptor's energy landscape.
	
	\subsection{Structural Insights and Energy Landscape Analysis}
	\textbf{Comment:} Explain how structural dynamics are integrated into the kinetic model and how this helps explain the energy landscape.
	
	Incorporating structural dynamics into the kinetic model allows us to analyze the energy landscape of P2X2 activation. We use a combination of molecular dynamics simulations and Bayesian inference to explore how ATP binding and channel opening affect the energy profile. The analysis reveals new insights into the energetic barriers governing subunit rotation and how these barriers influence the receptor's activation kinetics.
	
	\section{Results}
	\textbf{Comment:} Present the findings in a clear and concise manner, with visual aids (figures) to support the results.
	
	\subsection{Bayesian Inference of Kinetic Parameters}
	\textbf{Comment:} Report the results of the Bayesian model fitting, emphasizing the accuracy of the posterior distributions for kinetic parameters.
	
	We applied the Bayesian framework to the experimental data and obtained posterior distributions for the key kinetic parameters. Our results show a high degree of confidence in the estimated rates of ATP binding, channel opening, and closing. The inferred kinetic parameters are consistent with the expected behavior of P2X2, validating the model and the use of Bayesian inference in this context.
	
	\subsection{Energy Landscape and Structural Mechanisms}
	\textbf{Comment:} Discuss the results from the energy landscape analysis, highlighting any new insights into the structural mechanisms underlying receptor activation.
	
	The energy landscape analysis reveals a reduction in the activation barrier upon ATP binding, suggesting that ATP binding facilitates subunit rotation. This finding provides new structural insights into the activation mechanism of P2X2, indicating that the transition from a closed to an open state is energetically facilitated by allosteric interactions between subunits.
	
	\section{Discussion}
	\textbf{Comment:} In this section, interpret the results, discuss their implications, and highlight the novelty of your work.
	
	\subsection{Linking Kinetics to Structure}
	\textbf{Comment:} Discuss the significance of linking kinetic parameters to structural insights, emphasizing the novelty of your work in bridging these traditionally separate fields.
	
	Our work represents a significant step forward in linking kinetic modeling with structural insights. By incorporating aspects of molecular dynamics into our Bayesian framework, we have shown how changes in the kinetic parameters of P2X2 activation can be interpreted in terms of the receptor's energy landscape and structural transitions. This approach bridges the gap between kinetic and structural biology, offering a new way to study ion channel activation.
	
	\subsection{Implications for Ion Channel Pharmacology}
	\textbf{Comment:} Address the broader implications of your findings, particularly for drug development and the study of ion channels.
	
	Understanding the energy landscape of P2X2 activation provides valuable insights into how ion channels function and how they might be targeted in drug development. By improving our understanding of the activation mechanism, we can design better therapeutic strategies for modulating P2X2 receptors in various physiological contexts.
	
	\section{Conclusion}
	\textbf{Comment:} Summarize the main findings and their broader implications.
	
	This study introduces a novel Bayesian framework that bridges the gap between kinetic modeling and structural insights into P2X2 receptor activation. By linking experimental data to molecular dynamics, we provide new insights into the activation mechanism of P2X2 and highlight the potential of this approach for future studies of ion channels. Our findings have broad implications for both basic science and drug development.
	
	\section*{Acknowledgments}
	\textbf{Comment:} Acknowledge funding sources, collaborators, and any other contributions.
	
	We would like to thank [funding sources] for their support and [collaborators] for their helpful discussions. This research was supported by [grant number or funding agency].
	
\end{document}
