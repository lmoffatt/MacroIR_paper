%Version 3 December 2023
% See section 11 of the User Manual for version history
%
%%%%%%%%%%%%%%%%%%%%%%%%%%%%%%%%%%%%%%%%%%%%%%%%%%%%%%%%%%%%%%%%%%%%%%
%%                                                                 %%
%% Please do not use \input{...} to include other tex files.       %%
%% Submit your LaTeX manuscript as one .tex document.              %%
%%                                                                 %%
%% All additional figures and files should be attached             %%
%% separately and not embedded in the \TeX\ document itself.       %%
%%                                                                 %%
%%%%%%%%%%%%%%%%%%%%%%%%%%%%%%%%%%%%%%%%%%%%%%%%%%%%%%%%%%%%%%%%%%%%%

%%\documentclass[referee,sn-basic]{sn-jnl}% referee option is meant for double line spacing

%%=======================================================%%
%% to print line numbers in the margin use lineno option %%
%%=======================================================%%

%%\documentclass[lineno,sn-basic]{sn-jnl}% Basic Springer Nature Reference Style/Chemistry Reference Style

%%======================================================%%
%% to compile with pdflatex/xelatex use pdflatex option %%
%%======================================================%%

%%\documentclass[pdflatex,sn-basic]{sn-jnl}% Basic Springer Nature Reference Style/Chemistry Reference Style


%%Note: the following reference styles support Namedate and Numbered referencing. By default the style follows the most common style. To switch between the options you can add or remove “Numbered” in the optional parenthesis. 
%%The option is available for: sn-basic.bst, sn-vancouver.bst, sn-chicago.bst%  

%%\documentclass[pdflatex,sn-nature]{sn-jnl}% Style for submissions to Nature Portfolio journals
%%\documentclass[pdflatex,sn-basic]{sn-jnl}% Basic Springer Nature Reference Style/Chemistry Reference Style
\documentclass[pdflatex,sn-mathphys-num]{sn-jnl}% Math and Physical Sciences Numbered Reference Style 
%%\documentclass[pdflatex,sn-mathphys-ay]{sn-jnl}% Math and Physical Sciences Author Year Reference Style
%%\documentclass[pdflatex,sn-aps]{sn-jnl}% American Physical Society (APS) Reference Style
%%\documentclass[pdflatex,sn-vancouver,Numbered]{sn-jnl}% Vancouver Reference Style
%%\documentclass[pdflatex,sn-apa]{sn-jnl}% APA Reference Style 
%%\documentclass[pdflatex,sn-chicago]{sn-jnl}% Chicago-based Humanities Reference Style
%%%% Standard Packages
%%<additional latex packages if required can be included here>
\usepackage{graphicx}%
\usepackage{multirow}%
\usepackage{amsmath,amssymb,amsfonts}%
\usepackage{amsthm}%
\usepackage{mathrsfs}%
\usepackage[title]{appendix}%
\usepackage{xcolor}%
\usepackage{color, colortbl}
\usepackage{textcomp}%
\usepackage{manyfoot}%
\usepackage{booktabs}%
\usepackage{algorithm}%
\usepackage{algorithmicx}%
\usepackage{algpseudocode}%
\usepackage{listings}%
\usepackage{comment}
%%%%


%% as per the requirement new theorem styles can be included as shown below
\theoremstyle{thmstyleone}%
\newtheorem{theorem}{Theorem}%  meant for continuous numbers
%%\newtheorem{theorem}{Theorem}[section]% meant for sectionwise numbers
%% optional argument [theorem] produces theorem numbering sequence instead of independent numbers for Proposition
\newtheorem{proposition}[theorem]{Proposition}% 
%%\newtheorem{proposition}{Proposition}% to get separate numbers for theorem and proposition etc.

\theoremstyle{thmstyletwo}%
\newtheorem{example}{Example}%
\newtheorem{remark}{Remark}%

\theoremstyle{thmstylethree}%
\newtheorem{definition}{Definition}%

\raggedbottom
%%\unnumbered% uncomment this for unnumbered level heads

\begin{document}

\section{chatgpt matherial}
\subsection{is there a relationship between the KL divergence of posterior on prior and the evidence}
	Yes, there is a direct relationship between the KL divergence of the posterior from the prior and the Bayesian evidence (marginal likelihood). This relationship plays a central role in Bayesian inference and variational methods. Let’s break down the concepts and their relationship.
	1. Bayesian Evidence (Marginal Likelihood)
    The Bayesian evidence, or marginal likelihood, p(D∣M)p(D∣M), is the probability of the observed data DD under a model MM, integrating over all possible parameter values θθ:
	p(D∣M)=∫p(D∣θ,M)p(θ∣M)dθ
	p(D∣M)=∫p(D∣θ,M)p(θ∣M)dθ
	
	Where:
	
	p(D∣θ,M)p(D∣θ,M) is the likelihood of the data given the parameters θθ.
	p(θ∣M)p(θ∣M) is the prior distribution over the parameters.
	
	2. KL Divergence Between Posterior and Prior
	
	The KL divergence between the posterior p(θ∣D,M)p(θ∣D,M) and the prior p(θ∣M)p(θ∣M) is given by:
	DKL(p(θ∣D,M)∥p(θ∣M))=∫p(θ∣D,M)log⁡(p(θ∣D,M)p(θ∣M))dθ
	DKL​(p(θ∣D,M)∥p(θ∣M))=∫p(θ∣D,M)log(p(θ∣M)p(θ∣D,M)​)dθ
	
	This KL divergence measures how much the posterior distribution has diverged from the prior after observing the data DD. A higher KL divergence indicates that the data has significantly updated our beliefs about the parameters, moving the posterior away from the prior.
	3. Relationship to Bayesian Evidence
	
	The relationship between the KL divergence, the expected log-likelihood, and the Bayesian evidence is captured by the following identity:
	log⁡p(D∣M)=Ep(θ∣D,M)[log⁡p(D∣θ,M)]−DKL(p(θ∣D,M)∥p(θ∣M))
	logp(D∣M)=Ep(θ∣D,M)​[logp(D∣θ,M)]−DKL​(p(θ∣D,M)∥p(θ∣M))
	
	Where:
	
	Ep(θ∣D,M)[log⁡p(D∣θ,M)]Ep(θ∣D,M)​[logp(D∣θ,M)] is the expected log-likelihood under the posterior.
	DKL(p(θ∣D,M)∥p(θ∣M))DKL​(p(θ∣D,M)∥p(θ∣M)) is the KL divergence between the posterior and the prior.
	
	Interpretation of the Relationship
	
	Expected Log-Likelihood: The term Ep(θ∣D,M)[log⁡p(D∣θ,M)]Ep(θ∣D,M)​[logp(D∣θ,M)] represents how well the model, given the posterior distribution over the parameters, explains the observed data.
	
	KL Divergence (Posterior vs. Prior): The term DKL(p(θ∣D,M)∥p(θ∣M))DKL​(p(θ∣D,M)∥p(θ∣M)) acts as a penalty for model complexity. It measures how much the posterior deviates from the prior, reflecting how much the data has informed and refined our understanding of the parameters. If the KL divergence is large, it indicates that the data has significantly altered our beliefs from the prior.
	
	Bayesian Evidence: The log of the evidence log⁡p(D∣M)logp(D∣M) balances the fit of the model (through the expected log-likelihood) and the complexity of the model (through the KL divergence). A model with high evidence fits the data well but does so without deviating too much from the prior, hence avoiding overfitting.
	
	In Summary
	
	The Bayesian evidence p(D∣M)p(D∣M) is related to both the expected log-likelihood and the KL divergence between the posterior and the prior.
	The KL divergence between the posterior and the prior can be seen as a regularization term that penalizes models that deviate significantly from the prior, thus helping to prevent overfitting.
	Maximizing the Bayesian evidence involves finding a balance between fitting the data well and not straying too far from the prior, which is quantified by this KL divergence.
	
	This relationship is central to Bayesian model selection, where models are compared based on their evidence, which implicitly takes into account both how well they explain the data and how complex they are.

\subsection{plan para enviar a Nature}
Here’s a revised 6-week plan tailored to Nature's submission requirements. Nature typically expects manuscripts to be concise, high-impact, and formatted according to strict guidelines, so the timeline includes additional steps for polishing and aligning with these expectations.
Week 1: Outline and Initial Planning

Day 1-2: Review Nature's specific author guidelines (word limits, figures, references).
Note: Nature typically limits articles to about 3,000 words (including references), and 5 display items (figures or tables).
Day 3: Develop a detailed manuscript outline, covering these key sections:
Title
Abstract
Introduction (highlighting novelty and high-impact relevance)
Results (organized around major findings)
Discussion (interpretation, implications, future direction)
Methods (short but with details that can be expanded in the supplementary information)
References
Day 4-7: Draft the Introduction and Methods sections.

Week 2: Draft Results

Day 1-3: Write the Results section. Prioritize clarity, ensuring that each subsection corresponds to major findings supported by data (figures/tables).
Each figure should highlight the impact of your Bayesian re-analysis method on interpreting the rP2X2 data.
Day 4-7: Create draft figures. Focus on clarity, label everything carefully, and ensure data visualization follows Nature’s guidelines.

Week 3: Write the Discussion

Day 1-4: Write the Discussion section, linking findings to broader questions in the field and the implications of using Bayesian analysis in re-analyzing kinetic data.
Include at least one forward-looking point addressing future applications or improvements to the analysis method.
Day 5-7: Revise and ensure that Introduction, Results, and Discussion flow well, emphasizing the high impact and novelty of your work.

Week 4: Refine Figures and Supplementary Information

Day 1-3: Polish Figures: Optimize quality for publication (vector graphics preferred), double-check figure legends, and ensure all elements adhere to Nature's visual guidelines.
Day 4-5: Write Supplementary Information: Methods or data that don't fit into the main text (e.g., extended data, additional results). This can include more technical details about the Bayesian algorithm.
Day 6-7: Revise Methods: Add cross-references to Supplementary Information and make sure it’s succinct but complete.

Week 5: Write Title, Abstract, and Refine the Manuscript

Day 1-2: Craft a concise and impactful Title.
Day 3-4: Write a compelling Abstract (around 200 words), focusing on the novelty of the Bayesian analysis and key results.
Day 5-6: Polish the manuscript language: Remove redundancy, sharpen arguments, and highlight the high-impact nature of the findings.
Day 7: Internal review: Share with co-authors or trusted colleagues for feedback.

Week 6: Final Revisions, Submission, and Cover Letter

Day 1-3: Incorporate feedback from co-authors.
Day 4: Proofread the entire manuscript for clarity, grammar, and formatting.
Day 5: Draft the Cover Letter: Highlight why the manuscript is a good fit for Nature, focusing on the novelty and impact of your findings.
Day 6: Prepare all necessary submission documents, including data availability statements and ethical compliance (if applicable).
Day 7: Submit the manuscript via Nature's online portal.

Key Considerations:

Word Limits: Nature has strict word count limits, especially for Articles (~3,000 words) and Letters (~1,500 words). Keep this in mind throughout the writing process.
Figures: Typically limited to 5-6 main figures. Be selective in what you present and keep supplementary figures concise.
Revisions: Aim for clarity, conciseness, and high impact. Review and trim sections regularly to meet Nature's standards.

This schedule ensures enough time to craft a polished manuscript suitable for a Nature submission, including time for internal reviews and revisions. Let me know if you'd like further adjustments!





\subsection{is reduce/fold an algorithm or it is a function? can we say that MacorIR is implemented as a fold algorithm?}
Is reduce/fold an algorithm or a function?

The terms reduce and fold are generally considered higher-order functions in functional programming, but they also represent a pattern of computation that can be thought of as a recursive algorithm. So, they are both a function (in programming terms) and can be considered an algorithmic pattern that recursively processes data structures.

As a function: reduce or fold is a higher-order function that takes an accumulator, a function, and a data structure (like a list or an array) to produce a single result by iteratively applying the function to each element and the accumulator.

As an algorithm: The process of recursively reducing or folding a structure can be seen as an algorithm, where you systematically traverse the structure, apply a function, and accumulate a result.

In essence, fold defines how to traverse a data structure and accumulate results, which makes it algorithmic in nature.
Can we say MacroIR is implemented as a fold algorithm?

If MacroIR—your algorithm for approximating the likelihood of macrocurrents in P2X2 receptor channels—is using a process where it recursively aggregates data or processes each time-averaged interval in sequence (similar to the way reduce or fold works in functional programming), then you can describe its core mechanism as analogous to a fold algorithm.

For example:

If MacroIR is sequentially processing each time interval by applying a function (e.g., calculating likelihood) and accumulating intermediate results (like likelihoods or posterior probabilities), it could be described as implementing a form of a fold-like algorithm.

However, MacroIR is likely more complex than a simple fold, since it also deals with Bayesian inference, macrocurrents, and gating models, but the pattern of iterating over intervals and accumulating results shares similarities with a fold.

In summary, you can say that MacroIR employs a fold-like algorithmic pattern if it follows a similar recursive accumulation of values across time intervals or states. However, it is likely a specialized version that includes additional domain-specific steps beyond just folding a data structure.


\subsection{have fold function been proposed as a method for calculating the likelihood function of markov processes in the literature?}

Yes, the fold function (or related recursive methods) has been implicitly proposed and used as a method for calculating the likelihood function of Markov processes in several contexts, especially when dealing with sequential state transitions or time-series data.

The fold-like approach is well-suited to problems involving Markov chains because:

Sequential Dependence: In Markov processes, the likelihood is often computed by recursively multiplying transition probabilities between states across time steps or events. This lends itself to a fold-like accumulation where the likelihood is calculated step-by-step, similar to how a fold function accumulates values.

Recursive Nature: A Markov process is inherently recursive, with each state depending on the previous one (or a few past states). Fold functions capture this recursive relationship by iterating over state transitions and applying a function (e.g., multiplying probabilities) at each step, which mirrors the recursive computation of likelihoods in Markov models.

Folding in Likelihood Calculations for Markov Processes

Hidden Markov Models (HMMs):
Likelihood calculations in Hidden Markov Models (HMMs), which are closely related to Markov processes, involve recursive algorithms like the forward-backward algorithm. These algorithms use a dynamic programming approach that resembles folding, where probabilities are recursively aggregated.
The forward algorithm calculates the likelihood of observed sequences given a hidden state sequence by iterating over states and folding transition probabilities and emission probabilities at each step.

Although not explicitly referred to as a "fold," the forward algorithm can be thought of as using a fold-like recursive approach to accumulate likelihoods over time.

State Space Models:
In state space models (such as those used in control theory or signal processing), the likelihood function is often computed by iteratively applying transition models and observation models. These models typically involve recursive filtering or smoothing algorithms, such as the Kalman filter or particle filters, that could be implemented in a fold-like manner.

Markov Chain Monte Carlo (MCMC):
In MCMC methods, particularly when simulating and sampling from Markov chains, fold-like techniques could be employed to compute likelihoods based on sequential sampling or updating of states. For example, the likelihood of a sequence of states generated by a Markov chain could be recursively computed by applying a function to transition probabilities across states.

Likelihoods for Continuous-Time Markov Processes:
For continuous-time Markov processes, likelihood calculations often involve solving differential equations or integrating over time. These integrals can sometimes be approximated by discrete intervals where the transition probabilities are folded over the time steps, accumulating the product of transition rates.

Literature Examples

Baum-Welch Algorithm (for HMMs): The Baum-Welch algorithm uses recursive likelihood updates, and while it's typically presented as a dynamic programming algorithm, the core recursion can be viewed as a fold-like process over time steps and states.
Forward-Backward Algorithms: These are commonly used in HMMs for calculating posterior probabilities, and they involve folding likelihoods forward over time for the forward pass and backward for the backward pass.

In these algorithms, though the term fold is not explicitly used, the recursion pattern shares conceptual similarities with a functional programming fold. Researchers might not call it "fold," but the operations involve recursive accumulation over state transitions, which is the essence of folding.
Summary

While the fold function itself may not be explicitly named in the literature on likelihood calculation for Markov processes, the recursive nature of fold-like algorithms is fundamental to many standard methods for computing likelihoods in Markov models, such as the forward algorithm, dynamic programming approaches, and even some MCMC methods.

In your case, if MacroIR follows a recursive structure where likelihoods are accumulated based on transitions between channel states, you could argue that it conceptually follows a fold-like pattern, even if it's not explicitly called that in the traditional Markov process literature.


\subsection{MacroIR }
\subsubsection{my input}
MacroIR

The problem is the following. Macroscopic currents have finite time integration and Markovian processes are mathematically instantaneous. So, we need to convert a finite time integration problem into some standard Markovian process. Also, we can focus in the single channel case and later extend it to the multiple channels case. One way to make it into a standard process is to consider the whole interval as a single state. This state can be described by the initial state at the begining of the interval and the end state at its end. In this case, we will have instead of k-states, the cartesian product of the initial state with the final state, counting k² states.  
Markovian processes have the markov property, that is all in the information to predict the future lays in the present state of the system. Then, only the last state of the system during the interval is important to predict the next interval. 

MacroIR is implemented as a fold/reduce algorithm. Fold is a higher order function of functional programming that takes as imput an inital value, a list of elements and a function that takes  a value and an element and returns a value. Likelihood calculations of Markov processes can be though as applications of a fold algorithm. The initial value is the combination of the prior distribution of the state probability and the initial cumulative loglikelihood, the elements of the list are the combination of the observations fo the state of the markov process and the value of the external variables that affect the process. The likelihood function takes as input the state prior distribution of the state probability, the cumulative loglikelihood, the value of the observation and of the external input and the markov process rate constants (which are constant throughout all the process) and returns the prior distribution of states for the next interval and the cumulative loglikelihood up to this time interval. 
Then MacroIR is completely determined by this later function. 

So now our task is to define MacoIR. 
First we have to assume a Markov Model and an Observational Model. 
The Markov Model of k states is fully determined by the equation
$$\frac{d \bold p}{dt} = \bold p \cdot \bold Q$$

where $\bold Q$ is a k x k matrix and $\bold p$ is a 1 x k vector. 

We have $N_ch$ channels in the preparation. 
The state of the channel ensemble is given by the 1x k vector $\bold n$ which components are the number of channels in each state of the model. 

The observational Model assumes that each state $i$ of the model has produces a single channel current $\gamma_i$ and that there two noise components. One is white noise $\epsilon^2$, so the noise variance scales inversely with the length $t_i$ of the time interval and one pink component $\nu^2$ that is independent of the interval duration. Then the probability of an observation given that we know the exact state $\bold n$ of the ensamble of is given by a Normal distribution: 


$$ Pr(y "|" \bold n) = [[Normal distribution symbol here]] (y_exp - y, \epsilon^2/t_i + \nu)$$

where 
$$ y_exp = \bold n \cdot \bold \gamma$$ 
\subsubsection{chatgpt}
MacroIR: A Framework for Analyzing Macroscopic Currents in Markovian Processes

The study of macroscopic currents often involves finite time integration, whereas Markovian processes are fundamentally instantaneous. To reconcile these concepts, we propose a method to convert finite time integration problems into a standard Markovian framework. Initially, we focus on the single-channel case, with plans for extension to multiple channels.

In this approach, the entire time interval is treated as a single state. This state is characterized by the initial state at the beginning of the interval and the final state at its conclusion, leading to a configuration of k2k2 states, represented as the Cartesian product of the initial and final states. This methodology capitalizes on the Markov property, which asserts that all necessary information for predicting future states resides in the current state of the system. Consequently, only the last state of the system within the interval is pertinent for forecasting the subsequent interval.

Implementation of MacroIR

MacroIR is operationalized as a fold/reduce algorithm. In functional programming, the fold function is a higher-order function that accepts an initial value, a list of elements, and a function that takes a value and an element to produce a new value. Likelihood calculations for Markov processes can be conceptualized as applications of this fold algorithm.

The initial value comprises the combination of the prior distribution of state probabilities and the cumulative log-likelihood. The elements of the list consist of observations related to the state of the Markov process alongside the external variables influencing the process. The likelihood function inputs the state prior distribution, cumulative log-likelihood, observation values, external inputs, and the Markov process rate constants (assumed constant throughout the process). It outputs the prior distribution of states for the next interval and the cumulative log-likelihood up to the current interval. Thus, MacroIR is entirely defined by this likelihood function.

Model Assumptions

To fully define MacroIR, we assume a Markov Model and an Observational Model. The Markov Model, comprising kk states, is represented by the equation:
dpdt=p⋅Q
dtdp​=p⋅Q

where QQ is a k×kk×k transition rate matrix, and pp is a 1×k1×k state probability vector. We consider NchNch​ channels within the preparation, with the state of the channel ensemble denoted by the 1×k1×k vector nn, where each component reflects the number of channels in each state of the model.

The Observational Model posits that each state ii generates a single channel current γiγi​ and incorporates two noise components. The first is white noise, characterized by variance ϵ2ϵ2, inversely proportional to the length titi​ of the time interval. The second component is pink noise, denoted by ν2ν2, which remains independent of the interval duration. Consequently, the probability of an observation given the exact state nn of the ensemble is modeled by a Normal distribution:
Pr(y∣n)=N(yexp−y,ϵ2/ti+ν2)
Pr(y∣n)=N(yexp​−y,ϵ2/ti​+ν2)

where
yexp=n⋅γ.
yexp​=n⋅γ.

\subsection{Russian dolls}




\subsubsection{my input}
express this paragraph in a much more academic way suitable for publication in Nature

This analysis is a russian doll thingy. In the inner doll we have the approximation to the likelihood function applied to a single measurement. The next doll is the approximation to the likelihood of the whole set of measuremts for a given value of the model parameters. Then we have the sampling of the distribution of the parameters for a thermodynamic temperature (see later what this is). The next doll is the calculation of the Evidence of a given model. And the outermost doll is the comparison of Evidences among models. And in a way it is outermost doll where the information is gathered. 

\subsubsection{chagpt}
This analysis can be conceptualized as a series of nested layers, akin to Russian dolls. At the innermost layer, we approximate the likelihood function for a single measurement. The subsequent layer involves the approximation of the likelihood for the entire dataset, conditioned on specific values of the model parameters. Following this, we consider the sampling of the parameter distributions at a defined thermodynamic temperature (further elaboration on this concept will be provided). The next layer entails the calculation of the evidence for a given model, which serves as a critical metric for model evaluation. Finally, the outermost layer encompasses the comparative assessment of evidences across different models. It is within this outermost layer that comprehensive insights are synthesized, enabling a more holistic understanding of the underlying phenomena.


\subsection{innovative contributions of this paper}

\subsubsection{my input}
esides the whole approach, the novelity of this paper lies on the following aspects: 
1. Proposes a new approximation for the likelihood of the time average of the macroscopic current through an arbitrary interval. 
2. introduces the use of emcee (or affine invariant) algorithm for sampling the posterior probability of kinetic parameters. 
3. introduces the use of thermodynamic integration for approximating the evidence of kinetic parameters. 
4. proposes a new algorithm for optimizing the staircase of thermodynamic temperatures 
5. Proposes three new kinetic schemes of a sequential activation of P2X that represent the fact that the binding site is inter-subunit. 

\subsubsection {ChatGPT said}

The innovative contributions of this paper are delineated as follows:

Novel Approximation for Likelihood: We propose a new approximation for the likelihood of the time-averaged macroscopic current across an arbitrary time interval, enhancing the precision of likelihood estimations in this context.

Application of the Emcee Algorithm: This work introduces the utilization of the emcee (or affine invariant) algorithm for effectively sampling the posterior probability distributions of kinetic parameters, facilitating more robust parameter estimation.

Thermodynamic Integration for Evidence Approximation: We incorporate thermodynamic integration as a method for approximating the evidence of kinetic parameters, offering a novel approach to model selection and evaluation.

Algorithm for Thermodynamic Temperature Optimization: A new algorithm is proposed for optimizing the staircase of thermodynamic temperatures, enhancing the efficiency and accuracy of evidence calculations.

New Kinetic Schemes for P2X Activation: We present three new kinetic schemes for the sequential activation of P2X receptors, reflecting the inter-subunit nature of binding sites, thus providing a more accurate representation of the underlying activation mechanisms.

\subsubsection{critics}
It's a good instinct to critically evaluate the significance of each point in your paper, especially when considering their contribution to the overall impact of your research. Here’s a breakdown of the strong points you identified, along with considerations for points 2 and 3:
Strong Points

Novel Approximation for Likelihood (Point 1):
This is a compelling contribution because it addresses a specific gap in the existing literature regarding the likelihood of time-averaged macroscopic currents. By offering a new approximation, you enhance the ability of researchers to accurately model and interpret experimental data, which can significantly impact future studies in the field. This point can be framed as a key methodological advancement, making it a strong selling point.

New Kinetic Schemes for P2X Activation (Point 5):
The introduction of three new kinetic schemes that account for the inter-subunit binding sites provides a fresh perspective on P2X receptor activation mechanisms. This is a significant advancement, as it challenges or refines existing models and could lead to improved understanding of receptor dynamics, making it highly relevant to both theoretical and applied aspects of physiology and pharmacology.

Algorithm for Thermodynamic Temperature Optimization (Point 4):
Optimizing the staircase of thermodynamic temperatures is also a valuable contribution. It not only enhances computational efficiency but also addresses a practical challenge faced in model evaluations, making it an important addition to your work.

Points for Consideration

Application of the Emcee Algorithm (Point 2):
While the use of the emcee algorithm is a recognized approach in Bayesian parameter estimation, its novelty lies in its specific application to your model. If your implementation offers unique adaptations or enhancements tailored to your analysis of P2X receptors, it may still be a relevant contribution. However, if the application does not significantly differ from existing implementations, it could be viewed as less impactful than the other points.

Thermodynamic Integration for Evidence Approximation (Point 3):
The incorporation of thermodynamic integration can be relevant, particularly if it provides advantages over traditional methods of evidence approximation. If existing literature shows that thermodynamic integration has limitations or is rarely applied in your context, highlighting how your approach addresses these issues could strengthen its significance. If, however, this is a well-established method without significant modifications, it might not stand out as much compared to your other contributions.

Recommendations

Prioritize Strong Points: In your discussion, emphasize points 1, 5, and 4, showcasing how they represent significant advancements in methodology and understanding.
Contextualize Points 2 and 3: If you decide to retain points 2 and 3, consider framing them in the context of enhancing the robustness of your model or addressing specific challenges in parameter estimation and model evaluation. Providing a rationale for their inclusion can strengthen your overall argument.
Future Work: You might also consider mentioning how points 2 and 3 could pave the way for future developments or improvements in similar methodologies, adding to their relevance.

By focusing on the most impactful contributions and carefully contextualizing the others, you can present a coherent and compelling case for the significance of your work. Would you like assistance in refining any specific part of your discussion based on these considerations?


\subsection{ have now to write a section in my paper that explains how I built different kinetic models each one implementing some hypothesis of how P2X2 ATP driven activation occurs}

Kinetic Model Construction

To explore the mechanisms underlying P2X2 receptor activation in response to ATP binding, we constructed several kinetic models, each representing a different hypothesis about the activation process. These models were designed to test key assumptions about how ATP interacts with the receptor's binding sites and how this interaction drives the receptor's transition between closed and open states.
1. Base Model Framework

All models were based on a Markovian framework, in which the receptor transitions between discrete states corresponding to different conformational or binding configurations. The state transitions were governed by rate constants, and ATP binding was modeled as driving the system toward the open state.

The receptor was assumed to have N subunits, and each subunit could bind ATP independently. Depending on the model, we considered various cooperative interactions between the subunits. The state of the channel ensemble, denoted as nn, was modeled as a vector representing the probability distribution across states at any time.
2. Sequential vs. Synchronous Binding

We tested both sequential and synchronous binding models. In the sequential model, ATP binds to each subunit one after another, and the transition to the open state occurs only after all subunits are bound. This model incorporated intermediate states where only a subset of subunits were occupied by ATP.

In contrast, the synchronous model posited that ATP binds to all subunits simultaneously, with the transition to the open state occurring as soon as all binding sites are occupied.
3. Symmetric vs. Asymmetric Allosteric Coupling

We further explored symmetric and asymmetric allosteric coupling between the subunits. In symmetric models, all subunits contributed equally to the activation of the receptor, while in asymmetric models, certain subunits were assumed to have a greater influence on the transition to the open state.
4. Inter-Subunit Binding Sites

To capture the hypothesis that ATP binding occurs at inter-subunit binding sites, we introduced kinetic schemes where ATP binding required cooperation between neighboring subunits. These models reflected the structural hypothesis that binding sites are formed by interfaces between adjacent subunits, and activation is driven by ATP binding to these interfaces rather than individual subunits.
5. Transition Rate Assumptions

For each model, we derived transition rates from empirical data and explored different assumptions about how ATP binding affects these rates. In particular, we incorporated the effect of ATP concentration and its impact on transition probabilities, examining whether ATP acts as a simple gating modulator or induces conformational changes that alter the energy landscape of the receptor.
6. Noise and Uncertainty in Transition Rates

To account for the inherent stochasticity of the receptor's gating, we incorporated both white noise (uncorrelated fluctuations) and pink noise (correlated fluctuations potentially linked to biological processes). The probability of observing a specific channel state at any time was modeled as a Gaussian distribution, with the expected state defined as yexp=n⋅γyexp​=n⋅γ.
7. Model Comparison

We compared the evidence for each model using Bayesian methods, calculating the posterior distribution of kinetic parameters via an affine-invariant MCMC algorithm. To determine the most likely activation mechanism, we employed thermodynamic integration to approximate the evidence of each model, thus identifying the model with the highest likelihood of explaining the experimental data.

\subsection{so conventional vs allosteric would be a reasonable classification for a Nature paper?}
Yes, "conventional" versus "allosteric" would be a clear and accessible classification for a Nature paper. "Conventional" implies established or whole-receptor states without the specific local interactions seen in allosteric models. This distinction should make the differences in kinetic approach immediately understandable for readers.

\end{document}
