\documentclass[12pt]{article}

% Packages recommended for Nature Supplementary Information
\usepackage[margin=1in]{geometry}     % Set appropriate margins
\usepackage{setspace}                 % For adjustable line spacing
\usepackage{graphicx}                 % For figures
\usepackage{amsmath, amssymb}         % For equations and symbols
\usepackage{caption}                  % For caption formatting
\usepackage{float}                    % To precisely position tables/figures
\usepackage{lipsum}                   % For placeholder text (dummy text)
\usepackage{hyperref}                 % For hyperlinks
\usepackage{times}                    % Times font
\usepackage{booktabs} % For \toprule, \midrule and \bottomrule
\usepackage{siunitx} % Formats the units and values
\usepackage{pgfplotstable} % Generates table from .csv

\begin{document}
	
	% Title page with authors and addresses
	\begin{center}
		\Large\textbf{Supplementary Information}\\[1em]
		\normalsize
		\textbf{Title of the Supplementary Information}\\[1em]
		\textbf{Authors:} Luciano Moffatt\textsuperscript{1}, Gustavo Pierdominici-Sottile\textsuperscript{2}\\[1em]
		\small
		\textsuperscript{1} Instituto de Qu\'{i}mica F\'{i}sica de los Materiales, Medio Ambiente y Energ\'{i}a, Consejo Nacional de Investigaciones Científicas y T\'{e}cnicas, Facultad de Ciencia Exactas y Naturales , Universidad de Buenos Aires, Ciudad Aut\'{o}noma de Buenos Aires (1428), Argentina\\
		\textsuperscript{2} Departamento de Ciencia y Tecnolog\'{i}a, 
		Consejo Nacional de Investigaciones Científicas y T\'{e}cnicas,  Universidad Nacional de Quilmes, S\'{a}enz Pe\~{n}a 352,Bernal ( B1876BXD), Buenos Aires,Argentina\\
		
	\end{center}
	
	% Supplementary Methods
	\section*{Supplementary Methods}
	\label{sec:supp-methods}
	% Provide detailed methodology here.
	\lipsum[1] % Replace with your content
	
	% Supplementary Tables
	\section*{Supplementary Tables}
	\label{sec:supp-tables}
	% Example of a supplementary table:
	\begin{table}[H]
		\centering
		\caption{Supplementary Table 1: Description of parameters.}
		\begin{tabular}{lcc}
			\hline
			Parameter & Value & Description \\
			\hline
			Parameter 1 & 10 & Example description \\
			Parameter 2 & 20 & Example description \\
			\hline
		\end{tabular}
	\end{table}
	
	\begin{table}[h!]
		\begin{center}
			\caption{Autogenerated table from .csv file.}
			\label{table1}
			\pgfplotstabletypeset[
			col sep=comma, % the separator in our .csv file
			header=has colnames, % use the first row as column names
			columns={Algo,Sch,Proc,Dur,Iter,ESS,lnEv,CI,Rhat,Conv}, % specify the columns to display
			columns/Algo/.style={
				column type={l} % Left align text
			},
			columns/Sch/.style={
				column type={l} % Left align text
			},
			columns/Proc/.style={
				column type={l} % Left align text
			},
			columns/Dur/.style={
				column type={S},
				table-format=2.0 % Adjust format based on expected values
			},
			columns/Iter/.style={
				column name=Iterations, % Change display name
				column type={S},
				table-format=5.0 % Adjust format based on expected values
			},
			columns/ESS/.style={
				column type={S},
				table-format=5.0 % Adjust format based on expected values
			},
			columns/lnEv/.style={
				column type={S},
				table-format=-3.6 % Adjust format based on expected values
			},
			columns/CI/.style={
				column type={l} % Left align the interval
			},
			columns/Rhat/.style={
				column name=$R_{\text{hat}}$, % Use math mode for R_hat
				column type={S},
				table-format=1.6 % Adjust format based on expected values
			},
			columns/Conv/.style={
				column name=Conv,
				column type={c} % Center the convergence symbol
			},
			every head row/.style={
				before row={\toprule}, % have a rule at top
				after row={
					& & \multicolumn{1}{c}{ncores} & days & thousands & samples & ln of & 90\% & $R_{\text{hat}}$ & indication\\
					& & \multicolumn{1}{c}{processor number} & & of iterations & & Evidence & Credible interval & & of convergence\\
					\midrule} % rule under units
			},
			every last row/.style={after row=\bottomrule}, % rule at bottom
			]{Supplementary_table_1.csv} % filename/path to file
		\end{center}
	\end{table}\section*{Supplementary Discussion}
	\label{sec:supp-discussion}
	% Provide extended discussion and interpretations here.
	\lipsum[2] % Replace with your content
	
	% Supplementary Equations
	\section*{Supplementary Equations}
	\label{sec:supp-equations}
	% List and number your supplementary equations here.
	\begin{equation}
		E = mc^2
	\end{equation}
	\begin{equation}
		\nabla \cdot \mathbf{E} = \frac{\rho}{\varepsilon_0}
	\end{equation}
	
	% Supplementary Notes
	\section*{Supplementary Notes}
	\label{sec:supp-notes}
	% Include additional notes, clarifications (e.g., on statistical analyses), acknowledgements, and grant numbers here.
	\lipsum[3] % Replace with your content
	
\end{document}
